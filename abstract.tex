In the last twenty years, the existence of exoplanets (planets orbiting stars other than our own sun) has gone from conjecture to established fact.  The accelerating rate of exoplanet discovery has generated a wealth of important new knowledge, and is due mainly to the development and maturation of a large number of technologies that drive a variety of planet detection and observation methods.  The overall goal of the exoplanet community is to study planets around all types of stars, and across all ranges of planetary mass and orbit size.  With this capability we will be able to build confidence in planet formation and evolution theories and learn how our solar system came to exist.

Achieving this goal requires creating dedicated instrumentation capable of detecting signals that are a small fraction of the magnitude of signals we can observe today.  It also requires analyzing highly noisy data sets for the faint patterns that represent the presence of planets.  Accurate modeling and simulation are necessary for both these tasks.  With detailed planetary and observation models we can predict the type of data that will be generated when a specific instrument observes a specific planetary system.  This allows us to evaluate the performance of both the instrument and the data analysis methods used to extract planet signals from observational data.  The same simulations can help optimize observation scheduling and statistical analysis of data sets.  The purpose of this thesis is to lay down the groundwork necessary for building simulations of this type, and to demonstrate a few of their many possible applications.

First, we show how each of four different detection methods (astrometry, doppler spectroscopy, transit photometry and direct imaging) can be described using a common parameter set which also encodes sufficient information to propagate the described exosystem in time.  We analyze this parameter set and derive the distribution functions of several of its elements.  These are shown to be useful in calculating the probabilities of planetary detection.  Using this capability, we create a framework for simulating whole direct imaging planet-finding missions, incorporating detailed instrument models, observatory operations, and an automated algorithm for observation scheduling.  This framework is used in a series of case studies to evaluate the capabilities of multiple proposed missions.  Finally, we show how the same modeling framework used to generate the mission simulations can also be used, with the formalism of dynamic filtering, for data analysis and data set synthesis.