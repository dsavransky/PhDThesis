\chapter{Conclusions and Future Work}\label{ch:conclusions}

In order to gain a better understanding of how planets form and develop we must improve our ability to observe planets of all types, and to extract information from such observations.  The intent of this thesis was to create a unified formalism that could be applied towards both these goals.  To do so, it was necessary to identify the set of variables that could describe the time varying behavior of an exosystem, and map directly to the observables of any method that could be used to find and study planets.

Once this set of variables was identified, and mappings were generated to the various observables, we could begin to develop applications in support of our goals.  First, in order to improve the chances of future planet-finding instruments, we created a framework for simulating entire planet-finding surveys, taking into account instrument and observatory constraints.  Along the way, it was necessary to implement a rudimentary form of decision modeling so that entire timelines of observations could be scheduled automatically, without human interference.  The final utility that came from direct application of our instrument and exosystem models has proven to be an invaluable tool for evaluating proposed science instruments and survey plans.

Similarly,  by taking advantage of the optimal estimation formalism used for dynamic filtering, the very same models were brought to bear on the data analysis problem, and proved equally applicable.  The fact that this formalism is capable of dealing with multiple changing observation functions implies that we already have a proven method for synthesizing multiple data sets, which is certain to prove useful in the near future.

The applications presented in this thesis are only the first step.  As more detail is included in the simulations, they will become much more powerful tools for instrument evaluation and will help with setting up and running actual surveys.  More importantly, as our confidence in these simulations grows, we will be able to compare them with actual survey results and thus judge whether the underlying assumptions in the simulations (i.e., planetary populations or evolution theories) correspond to reality.  This final step will turn exoplanet finding from a blind search into a series of controlled experiments, and will greatly advance our knowledge of how we came to be here, and what the future has in store.